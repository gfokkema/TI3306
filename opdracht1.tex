\documentclass[11pt,twoside,a4paper]{article}
\usepackage[english]{babel}
\usepackage{amsmath}
\usepackage{amsthm}
\usepackage{amssymb}
\usepackage{graphicx}
\usepackage{hyperref}
\usepackage{multirow}
\usepackage{pbox}

%Pseudocode
\usepackage{algorithm}
\usepackage[noend]{algpseudocode}

%Automata
\usepackage{tikz}
\usetikzlibrary{automata,arrows}

\usepackage{a4wide,times}
\title{TI2316 \\
       Complexiteitstheorie \\
       Opdracht 1}
\author{Gerlof Fokkema, gfokkema, 4257286 \\
        Skip Lentz, smlentz, 4334051}
\begin{document}
\maketitle

Gegeven is het BOUNDED CIRCUIT probleem: \\

\begin{tabular}{ll}
\textbf{Instantie} & \pbox{20cm}{Een graaf $G = (V, E)$, voor iedere $e \in E$ een lengte $l(e) \in Z^+$ \\
                                 een subset $E' \subseteq E$ en een positief geheel getal $K$.} \\\\
\textbf{Vraag}     & \pbox{20cm}{Bestaat er een circuit in $G$ dat alle kanten in $E'$ bevat en een \\
                                 totale lengte heeft die $K$ of minder bedraagt?} \\
\end{tabular} \\\\


Om aan te tonen dat dit probleem een NP-moeilijk probleem is, wordt de volgende
transformatie van Hamiltoons Circuit naar dit probleem gegeven: \\

Laat G = (V, E) een instantie zijn van HAMILTOONS CIRCUIT.
Creeer de volgende instantie $(G_B = (V_B, E_B), l, E'_B, K)$ van BOUNDED CIRCUIT:
\par\noindent
\begin{flalign}
  V_B  &= V \cup \{ v'\ |\ v \in V \}
       & \\
  E_B  &= \{ \{ v, v' \}\ |\ v \in V \} \cup \{ \{v', w \}, \{ v, w' \}\ |\ \{ v, w \} \in E \}
       & \\
  E'_B &= \{ \{ v, v' \}\ |\ v \in V \}
       & \\
  l(e) &= 1 \text{ voor alle $e \in E_B$}
       & \\
  K    &= 2 \cdot |V|
       & 
\end{flalign}
\end{document}
